\section{Lean}

Lean is a new theorem proving environment from Microsoft Research.
\begin{itemize}
    \item powerful elaboration engine
    \item small trusted kernel
    \item flexible kernel
    \item incremental compilation
    \item universe polymorphism
    \item mixed declarative and tactic proof style
    \item powerful automation
\end{itemize}

The core ideas are:
\begin{itemize}
    \item type universes
    \item function spaces
    \item inductive types
\end{itemize}

\subsection{Type Theory}
Type theory is a foundational framework for mathematics. It was an approach that evolved out of Russel's work on setting
mathematics on a consistent foundational system for mathematics. The early 20th century was spent developing type theory
as a proof system. The key insight was the Curry-Howard Isomorphism, which roughly states that types can be associated with
propositions and proofs with programs. This insight was adapted and it became apparent that typed variants of the lambda
calculus can be used as proof systems, the most widely used of which is Martin-Lof type theory. You can then view types in
this typed lambda calculus as various features of constructive logic.

If we begin with just an implicational fragment of logic we can use the simply typed lambda calculus which only has a
single type constructor, the function type. This corresponds to implication. We can now view the typing judgment $a : A$
as statement that a is a proof of A, and we can interpret the typing judgment $f : A -> B$ as a computational implication.
At the type level it allows us to produce a B given an A, and computationally it gives us a method for transforming a program
of A into a program of B. If we remember that program's correspond to proofs we now have a computational method for proving.

We can of course extend our type system with richer features. We will quickly survey important additions to the Simply
Typed Lambda Calculus that are needed to turn it into a dependently typed lambda calculus. The key idea is the erasure
of phase distinction via the inclusion of a stratified type hierarchy, and Pi types.

Traditionally we maintain a phase distinction where types and values live in separate universes. Dependent types
are about easing this distinction. Type systems can usually be made more expressive by a series of additions.
The simplest of which is the introduction of quantifiers which allow us to abstract over types with type lambdas.
We can then introduce kinds (the types of types), and kind polymorphism which allow us to specify type shapes or (type arities).
We can continue to enrich our kind and type system, eventually reflecting types as kinds, and values as types,
but these systems are actually more complicated then a dependently typed calculi. Each of these refines become
more and more complex because it requires the language to maintain a discipline between values, kinds, sorts.
Dependent types simply introduce the idea of a Pi Type of dependent function space. This type system generalization is the
idea that we can make everything much simpler by creating a stratified hierarchy of types, and adding Pi and Inductive types.
We then can collapse our language into a simple set of pseudo terms, and typing rules which are easy to verify for correctness.

We can then ignore all the complexity introduced by types systems such as System F and generalizations of it. Instead we can
focus on a simple language. Usually a dependently typed core only requires a few forms abstraction, variables, function application,
and Pi Types. This is simply the lambda calculus that everyone knows enriched with Pi types. It is also useful to extend the calculus
with inductive types, or recursive data types that may be parametrized by the values they are constructed with.

Pi Types can be simply viewed as a generalization of the traditional function type. The generalization
allows the type of codomain to depend on value of the domain of the function. Given $B : P -> A; \Pi (x : A). B(x)$. This
simple extension actually gives rise to a lot of power and will allow us to encode much of mathematics in type theory.

This is of course a very brief overview and more information can be found in \cite{Pierce:TypeSystems} \cite{martinlof} and \cite{HoTTbook}.

\subsection{Theorem Proving}

Automated Theorem Proving has been an area of active research for many decades. It is apparent that the
verification of general properties is desirable to both high assurance software engineers, researches, and
mathematicians. There have been various approaches to Automated Theorem Proving in the past ACL2, NuPRL, LF,
LCF, Coq, Agda. The ones based on dependent type theories seem the most promising for various reasons
(computational, program extraction, proof carrying code, ect). Automated Theorem Proving has been applied in
many areas from the NTSB, NASA, JPL, and a variety of other places.

The current best way to leverage the proving power of Type Theory are Automated or Interactive Theorem Proving
environments. These allow for the automatic satisfaction of some properties, and interactive proving of others.
The state-of-the-art theorem proving environments in use for software verification are Coq, Agda, and Idris
all of which provide the benefits of:
\begin{itemize}
    \item dependent types (equivalent to a fragment of isolationistic logic)
    \item provides tools for automated proof search
    \item specification and implementation are one and the same
    \item executable "proofs" (i.e program extraction)
\end{itemize}

These are all based on versions of intensional type theory. Coq is base on
on Coquand and Huet's Theory of Constructions, and Agda is an evolution
Martin-Lof Type Theory \cite{martinlof}.

Automated Theorem Proving has been an area of active research for many decades. It is apparent that the
verification of general properties is desirable to both high assurance software engineers, researches, and
mathematicians. There have been various approaches to Automated Theorem Proving in the past ACL2, NuPRL, LF,
LCF, Coq, Agda. The ones based on dependent type theories seem the most promising for various reasons
(computational, program extraction, proof carrying code, ect). Th current state-of-art is either based
on Coquand and Huet's Theory of Constructions or the work done on Martin-Lof Type Theory \cite{martinlof}.
Automated Theorem Proving has been applied in many areas from the NTSB, NASA, JPL, and a variety of other places.

It appears that ideas from ATP could be very useful in the domain of hardware specification. Most
specification is performed separately from the implementation and suffers the common
problems of pen and paper proof. In many cases Embedded Domain Specific Languages (EDSLs)
have been used as a way to gain the power of a proof assistant and also write proof carrying code\cite{Ricketts:2014}\cite{fesi}. We want
to leverage the ideas here, but ideally built a separate front-end that exposes the full power of proving,
but also allows one to provide both an usable interface that also generates useful proof properties for
free \cite{Ricketts:2014}. \cite{chlipala2011certified} \cite{Pierce:SF}

There is a prototype implementation of ideas discussed in the final section available. It is currently
written in Idris (which is one of the more practical dependently typed programming languages).
