\section{Vrypto}

In this section we describe the current state of our prototype framework Vrypto
a portmanteau of Verified Crypto. Our framework is designed as an
embedded domain specific language of bits and sequences. We use a
a technique called parametric higher order abstract syntax in order
to represent variable binding, and we paramertrize our term language by both
the varible binder and type of expression, this means our programs are type
correct by construction, so any term we build is guarenteed to be correctly typed.

Our type system for the language currently supporting bits, sized sequences,
arbitrary sized integers and functions.

\begin{verbatim}
inductive ctype : Type :=
    | Unit     : ctype
    | Bit      : ctype
    | Sequence : forall (n : nat) (a : ctype), ctype
    | Fn       : ctype -> ctype -> ctype
\end{verbatim}

This type language seems simplistic but actually captures most of the interesting
behavior we need. We can easily define more complex types for example numbers
just become sequences of bits.

\begin{verbatim}
definition Number (n : nat) : ctype :=
    Sequence (divide n 2) a
\end{verbatim}

Our main language supports the definition and use of functions, streams, and bits.
We can use these simple pieces to write relatively complex programs, as well
as reason about program behavior. The encoding we use makes it possible to
define transforms simply as recursive functions over the tree structure. We
can also utilize proofs of equality to prove the equivalence of programs,
and thus algorithms.

\begin{verbatim}
variable {var : ctype -> Type}

inductive crypto : ctype -> Type :=
    | One  : crypto Bit
    | Zero : crypto Bit
    | Num  : forall {n}, crypto (Number n)
    | Empty : forall {c : ctype},
                crypto (Sequence zero c)
    | Cons : forall {n} {c : ctype},
                crypto c ->
                crypto (Sequence n c) ->
                crypto (Sequence (succ n) c)
    | Var  : forall {t}, var t -> crypto t
    | Abs  : forall {domain range},
                (var domain -> crypto range) ->
                crypto (Fn domain range)
    | App  : forall{domain range},
                crypto (Fn domain range) ->
                crypto domain ->
                crypto range
    | Let  : forall {t1 t2},
                crypto t1 ->
                (var t1 -> crypto t2) ->
                crypto t2
\end{verbatim}

Currently there are some unfinished parts of the proofs and we can not demonstrate
a complex example. When finished it would be possible to write statements like so:

\begin{verbatim}
example : (complexity algo1) = (complexity algo2)
\end{verbatim}

Where complexity is simply a recursive function that computes the complexity
class of an algorithm.
